% Emilio Rojas 2017

\documentclass[fontsize=8pt]{scrartcl}

% Emilio Rojas 2017

\usepackage{tikz}
\usepackage{float}

\usepackage{amsmath}

\usepackage{geometry}
\usepackage{multicol}

\usepackage{mathtools}

\usepackage{tikz-3dplot}

\usepackage[T1]{fontenc}
\usepackage[spanish]{babel}
\usepackage[utf8]{inputenc}


\geometry{
  margin=1cm,
  landscape
}
\allowdisplaybreaks
\setlength\columnsep{20pt}
\begin{document}
  \begin{multicols}{3}

    \begin{equation}
      \tag{*}
      F = k\frac{Q_1 Q_2}{R^2}
      \label{eqn:coulumb_simple}
    \end{equation}

    \begin{equation}
      \tag{*}
      k = \frac{1}{4\pi\epsilon_0}
      \label{eqn:coulomb_k}
    \end{equation}

    \begin{equation}
      \tag{Permitividad del espacio libre}
      \epsilon_0 = 8.854 \times 10^{-12} = \frac{1}{36\pi}10^{-9} \frac{\text{F}}{\text{m}}
      \label{eqn:permitividad_espacio_libre}
    \end{equation}

    \begin{equation}
      \tag{Ley de Coulumb Escalar}
      F=\frac{Q_1 Q_2}{4\pi\epsilon_0 R^2}
      \label{eqn:coulumb_escalar}
    \end{equation}

    \begin{equation}
      \tag{Ley de Coulumb Vectorial}
      \pmb{\text{F}}_2=\frac{Q_1 Q_2}{4\pi\epsilon_0 R^2_{12}} \pmb{\text{a}}_{12}
      \label{eqn:coulumb_vectorial}
    \end{equation}

    \begin{equation}
      \tag{Dirección entre 1 y 2}
      \pmb{\text{a}}_{12}=\frac{\pmb{\text{R}}_{12}}{\left|\pmb{\text{R}}_{12}\right|}
      \label{eqn:eps0}
    \end{equation}

  \end{multicols}
\end{document}
