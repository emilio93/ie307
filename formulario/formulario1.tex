% Emilio Rojas 2017

\documentclass[fontsize=8pt]{scrartcl}

% Emilio Rojas 2017

\usepackage{amsmath}

\usepackage{geometry}
\usepackage{multicol}

\usepackage{mathtools}

\usepackage[T1]{fontenc}
\usepackage[spanish]{babel}
\usepackage[utf8]{inputenc}

\input{../shared/commands.tex}

\geometry{
  margin=.7cm,
  landscape,
  legalpaper
}
\allowdisplaybreaks
\setlength\columnsep{20pt}
\begin{document}
  \begin{multicols}{3}

    \section*{Análisis Vectorial}

      \begin{equation*}
        \tag{magnitud}
        \left|\vec{B}\right| = \sqrt{B_X^2 + B_Y^2 + B_Z^2}
      \end{equation*}

      \begin{equation*}
        \tag{dirección}
        \vec{a}_B = \frac{\vec{B}}{\sqrt{B_X^2 + B_Y^2 + B_Z^2}} = \frac{\vec{B}}{\left|\vec{B}\right|}
      \end{equation*}

      \begin{equation*}
        \tag{prod. punto}
        \vec{A} \cdot \vec{B} = \left|\vec{A}\right| \left|\vec{B}\right| \cos(\theta_{AB})
      \end{equation*}

      \begin{equation*}
        \tag{prod. punto}
        \vec{A} \cdot \vec{B} = A_X B_X + A_Y B_Y + A_Z B_Z
      \end{equation*}

      \begin{equation*}
        \tag{prod. cruz}
        \vec{A} \times \vec{B} = \vec{a}_N \left|\vec{A}\right| \left|\vec{B}\right| \sen(\theta_{AB})
      \end{equation*}


      \begin{table}[H]
        \begin{tabular}{|c|}
          \hline \\
          $\vec{a}_x$\\
          $\vec{a}_y$\\
          $\vec{a}_z$\\\hline
        \end{tabular}
        \begin{tabular}{| c c c |}
          \hline
          $\vec{a}_\rho$ & $\vec{a}_\phi$  & $\vec{a}_z$ \\ \hline
          $\cos(\phi)$   & $-\sen(\phi)$   & $0$         \\
          $\sen(\phi)$   & $\cos(\phi)$    & $0$         \\
          $0$            & $0$             & $1$         \\ \hline
        \end{tabular}
        \begin{tabular}{| c c c |}
          \hline
          $\vec{a}_r$ & $\vec{a}_\theta$  & $\vec{a}_\phi$ \\ \hline
          $\sen(\theta)\cos(\phi)$ & $\cos(\theta)\cos(\phi)$ & $-\sen(\phi)$         \\
          $\sen(\theta)\sen(\phi)$ & $\cos(\theta)\sen(\phi)$  & $\cos(\phi)$         \\
          $\cos(\theta)$           & $-\sen(\theta)$             & $0$         \\ \hline
        \end{tabular}
      \end{table}

    \section*{Ley de Coulomb}

    \begin{equation*}
      \tag{permitividad del espacio}
      \epsilon_0 = 8.854 \times 10^{-12} = \frac{1}{36\pi} 10^{-9} \text{F}/\text{m}
    \end{equation*}

    \begin{equation*}
      \tag{fuerza eléctrica}
      \vec{F} = \frac{Q_1 Q_2}{4\pi\epsilon_0R_{12}^2} \vec{a}_{21} = -\frac{Q_1 Q_2}{4\pi\epsilon_0R_{12}^2} \vec{a}_{21}
    \end{equation*}

    \section*{Densidad de Flujo Eléctrico}

    \section*{Energía y Potencial}

    \section*{Corriente y Conductores}
  \end{multicols}
\end{document}
